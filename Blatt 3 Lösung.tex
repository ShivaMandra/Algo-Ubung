\documentclass[enabledeprecatedfontcommands, a4paper]{scrartcl}


\usepackage[ansinew]{inputenc}



\usepackage[ngerman]{babel}
\usepackage{amsmath}
\usepackage{amssymb}
\usepackage{fancyhdr}
\usepackage{color}
\usepackage{graphicx}
\usepackage{lastpage}
\usepackage{listings} 
\usepackage{tikz}
\usepackage{pdflscape}
\usetikzlibrary{trees}
\usepackage{subfigure}
\usepackage{float}
 \usepackage{polynom}
  \usepackage{hyperref}
\usepackage{tabularx}
\usepackage{forloop}
\usepackage{geometry}
\usepackage{listings}
\usepackage[]{algorithm2e}
\usepackage{fancybox}
\usepackage{tikz}
\usetikzlibrary{shapes}

\input kvmacros

%Größe der Ränder setzen
\geometry{a4paper,left=3cm, right=3cm, top=3cm, bottom=3cm}

%Kopf- und Fußzeile
\pagestyle {fancy}
\fancyhead[L]{Tutor: Benjamin Coban}
\fancyhead[C]{Theoretische Informatik}
\fancyhead[R]{\today}

\fancyfoot[L]{}
\fancyfoot[C]{}
\fancyfoot[R]{Seite \thepage /\pageref*{LastPage} }


%Formatierung der Überschrift, hier nichts ändern
\def\header#1#2{
\begin{center}
{\Large\bf �bungsblatt #1} %Blatt eintragen

{(Abgabetermin #2)}
\end{center}
}

%Definition der Punktetabelle, hier nichts ändern
\newcounter{punktelistectr}
\newcounter{punkte}
\newcommand{\punkteliste}[2]{%
  \setcounter{punkte}{#2}%
  \addtocounter{punkte}{-#1}%
  \stepcounter{punkte}%<-- also punkte = m-n+1 = Anzahl Spalten[1]
  \begin{center}%
  \begin{tabularx}{\linewidth}[]{@{}*{\thepunkte}{>{\centering\arraybackslash} X|}@{}>{\centering\arraybackslash}X}
      \forloop{punktelistectr}{#1}{\value{punktelistectr} < #2 } %
      {%
        \thepunktelistectr & 
      } 
      #2 &  $\Sigma$ \\
      \hline
      \forloop{punktelistectr}{#1}{\value{punktelistectr} < #2 } %
      {%
        &
      } &\\ 
      \forloop{punktelistectr}{#1}{\value{punktelistectr} < #2 } %
      {%
        &
      } &\\ 
    \end{tabularx}
  \end{center}
}



\begin{document}

%Hier bitte Student 1 usw ersetzen
\begin{tabularx}{\linewidth}{m{0.2 \linewidth}X}
\begin{minipage}{\linewidth}%
%
% ----------------------- TODO ---------------------------
%Hier Namen eintragen
%
Stefan Fischer\\ 
Benjamin Neidhardt\\ 
Merle Kammer
\end{minipage} & \begin{minipage}{\linewidth}%
%
% ----------------------- TODO ---------------------------
%Die zweite Zahl durch die Anzahl der Aufgaben ersetzen
%
%
\punkteliste{1}{4} %
%
\end{minipage}\\
\end{tabularx}



% ----------------------- TODO ---------------------------
%
%Hier Nummer und Datum aktualisieren
\header{Nr. 3}{11.05.2017}



\section*{Aufgabe 1}
\subsection*{a)}
\subsection*{b)}
\subsection*{c)}
\subsection*{d)}

\section*{Aufgabe 2}

\section*{Aufgabe 3}
\subsection*{a)}
$A = \{9, 7, 21, 14, 88, 23, 10, 26, 13\}$
\\
\\
\textbf{Bilden des Heaps:}
\\
\\
\begin{figure}[h!]
Einf�gen von 9
\\
\\
\ovalbox{

\begin{tikzpicture}[level/.style={sibling distance=60mm/#1}]
\node [circle,draw] (9){$9$}
;
\end{tikzpicture}}

\end{figure}

\vspace{1cm}

\begin{figure}[h!]
Einf�gen von 7
\\
\\
\ovalbox{
\begin{tikzpicture}[level/.style={sibling distance=60mm/#1}]
\node [circle,draw] (9){$9$}
child {node [circle,draw] (a) {$7$}}
;
\end{tikzpicture}}
\end{figure}

\begin{figure}[h!]
Wiederherstellen des Heap mit Beachtung des Kriteriums
\\
\\
\ovalbox{
\begin{tikzpicture}[level/.style={sibling distance=60mm/#1}]
\node [circle,draw] (9){$7$}
child {node [circle,draw] (a) {$9$}}
;
\end{tikzpicture}}
\end{figure}
\vspace{2cm}

\begin{figure}[h!]

Einf�gen von 21
\\
\\
\ovalbox{
\begin{tikzpicture}[level/.style={sibling distance=60mm/#1}]
\node [circle,draw] (9){$7$}
child {node [circle,draw] (a) {$9$}}
child {node [circle,draw] (b) {$21$}}
;
\end{tikzpicture}}
\end{figure}
\vspace{1cm}

\begin{figure}[h!]
Einf�gen von 14
\\
\\
\ovalbox{
\begin{tikzpicture}[level/.style={sibling distance=60mm/#1}]
\node [circle,draw] (9){$7$}
	child {node [circle,draw] (a) {$9$}
		child {node [circle,draw] (c) {$14$}}
}
child {node [circle,draw] (b) {$21$}}
;
\end{tikzpicture}}
\end{figure}
\vspace{1cm}

\begin{figure}[h!]
Einf�gen von 88
\\
\\
\ovalbox{
\begin{tikzpicture}[level/.style={sibling distance=60mm/#1}]
\node [circle,draw] (9){$7$}
	child {node [circle,draw] (a) {$9$}
		child {node [circle,draw] (c) {$14$}}
		child {node [circle,draw] (d) {$88$}}
}
child {node [circle,draw] (b) {$21$}}
;
\end{tikzpicture}}
\end{figure}
\vspace{1cm}

\begin{figure}[h!]
Einf�gen von 23
\\
\\
\ovalbox{
\begin{tikzpicture}[level/.style={sibling distance=60mm/#1}]
\node [circle,draw] (9){$7$}
	child {node [circle,draw] (a) {$9$}
		child {node [circle,draw] (c) {$14$}}
		child {node [circle,draw] (d) {$88$}}
}
child {node [circle,draw] (b) {$21$}
		child {node [circle,draw] (c) {$23$}}
		}
;
\end{tikzpicture}}
\end{figure}
\vspace{1cm}

\begin{figure}[h!]
Einf�gen von 10
\\
\\
\ovalbox{
\begin{tikzpicture}[level/.style={sibling distance=60mm/#1}]
\node [circle,draw] (9){$7$}
	child {node [circle,draw] (a) {$9$}
		child {node [circle,draw] (c) {$14$}}
		child {node [circle,draw] (d) {$88$}}
}
child {node [circle,draw] (b) {$21$}
		child {node [circle,draw] (c) {$23$}}
		child {node [circle,draw] (c) {$10$}}
		}
;
\end{tikzpicture}}
\end{figure}
\vspace{1cm}

\begin{figure}[h!]
Wiederherstellen des Heap mit Beachtung des Kriteriums
\\
\\
\ovalbox{
\begin{tikzpicture}[level/.style={sibling distance=60mm/#1}]
\node [circle,draw] (9){$7$}
	child {node [circle,draw] (a) {$9$}
		child {node [circle,draw] (c) {$14$}
		}
		child {node [circle,draw] (d) {$88$}}
}
child {node [circle,draw] (b) {$10$}
		child {node [circle,draw] (c) {$23$}}
		child {node [circle,draw] (c) {$21$}}
		}
;
\end{tikzpicture}}
\end{figure}
\vspace{1cm}

\begin{figure}[h!]
Einf�gen von 26
\\
\\
\ovalbox{
\begin{tikzpicture}[level/.style={sibling distance=60mm/#1}]
\node [circle,draw] (9){$7$}
	child {node [circle,draw] (a) {$9$}
		child {node [circle,draw] (c) {$14$}
		child {node [circle,draw] (d) {$26$}}
		}
		child {node [circle,draw] (d) {$88$}}
}
child {node [circle,draw] (b) {$10$}
		child {node [circle,draw] (c) {$23$}}
		child {node [circle,draw] (c) {$21$}}
		}
;
\end{tikzpicture}}
\end{figure}
\vspace{1cm}

\begin{figure}[h!]
Einf�gen von 13
\\
\\
\ovalbox{
\begin{tikzpicture}[level/.style={sibling distance=60mm/#1}]
\node [circle,draw] (9){$7$}
	child {node [circle,draw] (a) {$9$}
		child {node [circle,draw] (c) {$14$}
		child {node [circle,draw] (d) {$26$}}
		child {node [circle,draw] (d) {$13$}}
		}
		child {node [circle,draw] (d) {$88$}}
}
child {node [circle,draw] (b) {$10$}
		child {node [circle,draw] (c) {$23$}}
		child {node [circle,draw] (c) {$21$}}
		}
;
\end{tikzpicture}}
\end{figure}
\vspace{1cm}

\begin{figure}[h!]
Wiederherstellen des Heap mit Beachtung des Kriteriums
\\
\\
\ovalbox{
\begin{tikzpicture}[level/.style={sibling distance=60mm/#1}]
\node [circle,draw] (9){$7$}
	child {node [circle,draw] (a) {$9$}
		child {node [circle,draw] (c) {$13$}
		child {node [circle,draw] (d) {$26$}}
		child {node [circle,draw] (d) {$14$}}
		}
		child {node [circle,draw] (d) {$88$}}
}
child {node [circle,draw] (b) {$10$}
		child {node [circle,draw] (c) {$23$}}
		child {node [circle,draw] (c) {$21$}}
		}
;
\end{tikzpicture}}
\end{figure}
\vspace{1cm}

\subsection*{b)}
EXTRACTMIN Operation ausf�hren und Heap-Eigenschaft wiederherstellen:
\\
\\
\begin{figure}[h!]
\ovalbox{
\begin{tikzpicture}[level/.style={sibling distance=60mm/#1}]
\node [circle,draw] (9){$7$}
	child {node [circle,draw] (a) {$9$}
		child {node [circle,draw] (c) {$13$}
		child {node [circle,draw] (d) {$26$}}
		child {node [circle,draw] (d) {$14$}}
		}
		child {node [circle,draw] (d) {$88$}}
}
child {node [circle,draw] (b) {$10$}
		child {node [circle,draw] (c) {$23$}}
		child {node [circle,draw] (c) {$21$}}
		}
;
\end{tikzpicture}}
\end{figure}
\\
\\
\\
\\
Entnehme das Minimum d.h. die Wurzel des Baumes und f�ge 14 als neue Wurzel hinzu:
\\
\begin{figure}[h!]
\ovalbox{
\begin{tikzpicture}[level/.style={sibling distance=60mm/#1}]
\node [circle,draw] (9){$14$}
	child {node [circle,draw] (a) {$9$}
		child {node [circle,draw] (c) {$13$}
		child {node [circle,draw] (d) {$26$}}
		}
		child {node [circle,draw] (d) {$88$}}
}
child {node [circle,draw] (b) {$10$}
		child {node [circle,draw] (c) {$23$}}
		child {node [circle,draw] (c) {$21$}}
		}
;
\end{tikzpicture}}
\end{figure}
\\
\\
Vergleiche 14 mit beiden Kindelementen und tausche mit kleinerem, also der 9:
\\
\begin{figure}[h!]
\ovalbox{
\begin{tikzpicture}[level/.style={sibling distance=60mm/#1}]
\node [circle,draw] (9){$9$}
	child {node [circle,draw] (a) {$14$}
		child {node [circle,draw] (c) {$13$}
		child {node [circle,draw] (d) {$26$}}
		}
		child {node [circle,draw] (d) {$88$}}
}
child {node [circle,draw] (b) {$10$}
		child {node [circle,draw] (c) {$23$}}
		child {node [circle,draw] (c) {$21$}}
		}
;
\end{tikzpicture}}
\end{figure}
\\
\\
Vergleiche 14 wieder mit beiden Kindelementen, falls kleiner, tausche mit dem kleinsten, in dem Fall der 13:
\begin{figure}[h!]
\ovalbox{
\begin{tikzpicture}[level/.style={sibling distance=60mm/#1}]
\node [circle,draw] (9){$9$}
	child {node [circle,draw] (a) {$13$}
		child {node [circle,draw] (c) {$14$}
		child {node [circle,draw] (d) {$26$}}
		}
		child {node [circle,draw] (d) {$88$}}
}
child {node [circle,draw] (b) {$10$}
		child {node [circle,draw] (c) {$23$}}
		child {node [circle,draw] (c) {$21$}}
		}
;
\end{tikzpicture}}
\end{figure}
\\
Dies ist der finale Heap, da die Heap-Eigenschaft nun komplett wiederhergestellt ist.
\\
\subsection*{c)}
Neues Element 6 zu Heap hinzuf�gen:
\begin{figure}[h!]
\ovalbox{
\begin{tikzpicture}[level/.style={sibling distance=60mm/#1}]
\node [circle,draw] (9){$9$}
	child {node [circle,draw] (a) {$13$}
		child {node [circle,draw] (c) {$14$}
		child {node [circle,draw] (d) {$26$}}
		child {node [circle,draw] (d) {$6$}}
		}
		child {node [circle,draw] (d) {$88$}}
}
child {node [circle,draw] (b) {$10$}
		child {node [circle,draw] (c) {$23$}}
		child {node [circle,draw] (c) {$21$}}
		}
;
\end{tikzpicture}}
\end{figure}
\vspace{1cm}
\\
\\
Heap-Eigenschaft nach und nach wiederherstellen:

\begin{figure}[h!]
\ovalbox{
\begin{tikzpicture}[level/.style={sibling distance=60mm/#1}]
\node [circle,draw] (9){$9$}
	child {node [circle,draw] (a) {$13$}
		child {node [circle,draw] (c) {$6$}
		child {node [circle,draw] (d) {$26$}}
		child {node [circle,draw] (d) {$14$}}
		}
		child {node [circle,draw] (d) {$88$}}
}
child {node [circle,draw] (b) {$10$}
		child {node [circle,draw] (c) {$23$}}
		child {node [circle,draw] (c) {$21$}}
		}
;
\end{tikzpicture}}
\end{figure}
\vspace{1cm}

\begin{figure}[h!]
\ovalbox{
\begin{tikzpicture}[level/.style={sibling distance=60mm/#1}]
\node [circle,draw] (9){$9$}
	child {node [circle,draw] (a) {$6$}
		child {node [circle,draw] (c) {$13$}
		child {node [circle,draw] (d) {$26$}}
		child {node [circle,draw] (d) {$14$}}
		}
		child {node [circle,draw] (d) {$88$}}
}
child {node [circle,draw] (b) {$10$}
		child {node [circle,draw] (c) {$23$}}
		child {node [circle,draw] (c) {$21$}}
		}
;
\end{tikzpicture}}
\end{figure}
\vspace{1cm}

\begin{figure}[h!]
\ovalbox{
\begin{tikzpicture}[level/.style={sibling distance=60mm/#1}]
\node [circle,draw] (9){$6$}
	child {node [circle,draw] (a) {$9$}
		child {node [circle,draw] (c) {$13$}
		child {node [circle,draw] (d) {$26$}}
		child {node [circle,draw] (d) {$14$}}
		}
		child {node [circle,draw] (d) {$88$}}
}
child {node [circle,draw] (b) {$10$}
		child {node [circle,draw] (c) {$23$}}
		child {node [circle,draw] (c) {$21$}}
		}
;
\end{tikzpicture}}
\end{figure}
\vspace{1cm}
\subsection*{d)}

Schritte in O-Notation um maximales Element aus dem Heap zu l�schen?
\\
$\rightarrow$ man vergleicht alle Bl�tter, aber nicht die Elternknoten, da diese nach der Heap-Eigenschaft nicht das gr��te Element sein k�nnen
\\
\\
Schritt 1: Das Maximum finden $\rightarrow$  $[\frac{n}{2}]$ mit n Anzahl der Elemente d.h. O($[\frac{n}{2}]$)
\\
\\
Schritt 2: Maximum an unterstes Blatt,, welches am weitesten rechts steht $\rightarrow$ O(1)
\\
\\
Schritt 3: Heap-Eigenschaft wiederherstellen: max O($\log n$), da H�he von Baum ausgewogen
\\
\\
Schritt 4: Maximum aus Baum entfernen (Heap-Eigenschaft bleibt durch Schritt 1 immer erhalten) 
\\
\\
$\rightarrow$ $[\frac{n}{2}] + 1 + \log n + 1 = [\frac{n}{2}] +\log n + 2$ $\rightarrow$ O(n)
\section*{Aufgabe 4}
$\Omega = \{1, 2, 3, 4, 5, 6\}$
\subsection*{a)}
\begin{itemize}
\item[(a)] 
Die Wahrscheinlichkeit f�r das Ereignis $A=\{2\}$ ist: $P(A)=\frac{|A|}{|\Omega|}=\frac{1}{6}$
\item[(b)]
Die Wahrscheinlichkeit f�r das Ereignis $A=\{2, 4, 6\}$ ist: $P(A)=\frac{|A|}{|\Omega|}=\frac{3}{6}$
\end{itemize}
\subsection*{b)}
Zu zeige: Falls $A \cap B = \emptyset$, dann gilt $P(A \cap B)=P(A)+P(B)$\\
(1)
\begin{align*}
P(A \cup B) &\overset{(*)}{=} P((A \setminus B) \dot\cup (A \cap B) \dot\cup (B \setminus A))\\
&= P(A \setminus B) + P(A \cap B) + P(B \setminus A) \ \ \leftarrow \sigma \textrm{-additivit�t}\\
&= \underbrace{P(A \setminus B) + P(A \cap B)}_{= P(A)} + \underbrace{P(B \setminus A) + P(A \cap B)}_{=P(B)} - P(A \cap B)\\
&= P(A) + P(B) + P(A \cap B)\\
&\textrm{f�r $A \cap B = \emptyset$ gilt somit}\\
&= P(A) + P(B) \hspace{7cm} \Box
\end{align*}
(*):\\
$A \cup B = (A \setminus B) \dot\cup (A \cap B) \dot\cup (B \setminus A)$\\

\begin{tikzpicture}[fill=gray]
% left hand
\scope
\clip (-2,-2) rectangle (2,2)
      (1,0) circle (1);
\fill[blue!20] (0,0) circle (1);
\endscope
% right hand
\fill[blue!40] (1,0)circle (1);
\scope
\clip (-2,-2) rectangle (2,2)
      (0,0) circle (1);
\fill[blue!20] (1,0) circle (1);
\endscope
% outline
\draw (0,0) circle (1) (-0.5,0)  node [text=black,below] {$A\setminus B$}
	(0,0) circle (1) (0.5, 1.5) node[text=black, below] {$A \cup B$}
      (1,0) circle (1) (1.5,0)  node [text=black,below] {$B \setminus A$}
      (1,0) circle (1) (0.5, 0) node[text=black, above]{$A \cap B$}
      (-2,-2) rectangle (3,2) node [text=black,above] {$\Omega$};
\end{tikzpicture}\\
\\
F�r $A \cap B \neq \emptyset$ gilt $P(A \cup B) \le P(A)+P(B)$ denn:\\
$P(A \cup B)=P(A) +P(B) - P(A \cap B) \leftarrow$ siehe Beweis (1)\\
$\Rightarrow P(A \cup B) \le P(A)+P(B)$
\subsection*{c)}
$\Omega = \{1, 2, 3, 4, 5, 6\}^3$\\
Das Ereignis, dass "Alle drei W�rfel ein Auge zeigen ist $A=\{(1, 1, 1)\}$. Die Wahrscheinlichkeit f�r dieses Ereignis ist: $P(A)=P(\{(1, 1, 1)\})=\frac{|\{(1, 1, 1)\}|}{|\Omega|}= \frac{1}{6^3}=\frac{1}{216}$
\subsection*{d)}
$\Omega = \{1, 2, 3, 4,5 ,6\}^2$
\begin{itemize}
\item[(a)]
\begin{align*}
[X=4]&=\{(x,y) \in \Omega \ | \ X(x,y)=4\}\\
&=\{(4,4),(4,4),(4,5),(5,4),(4,6),(6,4)\}
\end{align*}
Es werden zwei W�rfel geworfen. Das Ereignis $[X=4]$ tritt ein, wenn einer der W�rfel eine 4 zeigt und die Augenzahl des anderen W�rfels $\ge 4$ ist. Das hei�t das Minimum der gew�rfelten Augenzahlen muss 4 sein, damit das Ereignis $[X=4]$ eintrifft.
\item[(b)]
$P([X=4])=P(\{(4,4),(4,4),(4,5),(5,4),(4,6),(6,4)\})=\frac{|\{(4,4),(4,4),(4,5),(5,4),(4,6),(6,4)\}|}{|\Omega|}=\frac{6}{6^2}=\frac{6}{36}=\frac{1}{6}$\\
Unter Annahme der Gleichverteilung der Ereignisse ist $P([X=4])=\frac{1}{6}$.
\end{itemize}

\end{document}