\documentclass[enabledeprecatedfontcommands, a4paper]{scrartcl}


\usepackage[ansinew]{inputenc}


\usepackage{fixltx2e}
\usepackage[ngerman]{babel}
\usepackage{amsmath}
\usepackage{amssymb}
\usepackage{fancyhdr}
\usepackage{color}
\usepackage{graphicx}
\usepackage{lastpage}
\usepackage{listings} 
\usepackage{tikz}
\usepackage{pdflscape}
\usetikzlibrary{trees}
\usepackage{subfigure}
\usepackage{float}
 \usepackage{polynom}
  \usepackage{hyperref}
\usepackage{tabularx}
\usepackage{forloop}
\usepackage{geometry}
\usepackage{listings}
\usepackage[]{algorithm2e}
\usepackage{fancybox}
\usepackage{tikz}
\usetikzlibrary{shapes}
\usepackage{listings} 
 \usepackage{color}   
 \usepackage[svgnames]{xcolor} 
x\lstset{
 	language=Java, 
    basicstyle={\small, \ttfamily},
    keywordstyle=\color{blue!80!black!100}, 
    identifierstyle=, 
    commentstyle=\color{green!50!black!100}, 
    stringstyle=\ttfamily, 
    breaklines=true, 
    numbers=none, 
    numberstyle=\small, 
    frame=tb, 
    backgroundcolor=\color{blue!3} ,
    aboveskip=3mm,
    belowskip=3mm,
    showstringspaces=false,
    columns=flexible,
    breaklines=trues,
    breakatwhitespace=true,
    tabsize=3
 } 

\input kvmacros

%Größe der Ränder setzen
\geometry{a4paper,left=3cm, right=3cm, top=3cm, bottom=3cm}

%Kopf- und Fußzeile
\pagestyle {fancy}
\fancyhead[L]{Tutor: Benjamin Coban}
\fancyhead[C]{Theoretische Informatik}
\fancyhead[R]{\today}

\fancyfoot[L]{}
\fancyfoot[C]{}
\fancyfoot[R]{Seite \thepage /\pageref*{LastPage} }


%Formatierung der Überschrift, hier nichts ändern
\def\header#1#2{
\begin{center}
{\Large\bf �bungsblatt #1} %Blatt eintragen

{(Abgabetermin #2)}
\end{center}
}

%Definition der Punktetabelle, hier nichts ändern
\newcounter{punktelistectr}
\newcounter{punkte}
\newcommand{\punkteliste}[2]{%
  \setcounter{punkte}{#2}%
  \addtocounter{punkte}{-#1}%
  \stepcounter{punkte}%<-- also punkte = m-n+1 = Anzahl Spalten[1]
  \begin{center}%
  \begin{tabularx}{\linewidth}[]{@{}*{\thepunkte}{>{\centering\arraybackslash} X|}@{}>{\centering\arraybackslash}X}
      \forloop{punktelistectr}{#1}{\value{punktelistectr} < #2 } %
      {%
        \thepunktelistectr & 
      } 
      #2 &  $\Sigma$ \\
      \hline
      \forloop{punktelistectr}{#1}{\value{punktelistectr} < #2 } %
      {%
        &
      } &\\ 
      \forloop{punktelistectr}{#1}{\value{punktelistectr} < #2 } %
      {%
        &
      } &\\ 
    \end{tabularx}
  \end{center}
}



\begin{document}

%Hier bitte Student 1 usw ersetzen
\begin{tabularx}{\linewidth}{m{0.2 \linewidth}X}
\begin{minipage}{\linewidth}%
%
% ----------------------- TODO ---------------------------
%Hier Namen eintragen
%
Stefan Fischer\\ 
Benjamin Neidhardt\\ 
Merle Kammer
\end{minipage} & \begin{minipage}{\linewidth}%
%
% ----------------------- TODO ---------------------------
%Die zweite Zahl durch die Anzahl der Aufgaben ersetzen
%
%
\punkteliste{1}{4} %
%
\end{minipage}\\
\end{tabularx}



% ----------------------- TODO ---------------------------
%
%Hier Nummer und Datum aktualisieren
\header{Nr. 4}{18.05.2017}



\section*{Aufgabe 1}
\subsection*{a)}
 \begin{lstlisting} 
smallest Difference (Array a){
	mergeSort (a);
		int minDiff = a[1] - a[0];
			int res = 0;
					for (int i = 1; i< a.lenght -1; i = i +1){
				int curDiff = a[i+1] - a[i];
			if (curDiff < minDiff)){
		minDiff = curDiff;
	res = i;}
	}
return res;}

 \end{lstlisting}\\
 

\noindent MergeSort hat bekannterweise eine Laufzeit von $O(nlogn)$ wenn wir jeweils zwei Teilfolgen mischen.
Nachdem wir sortiert haben wird jedem Schritt die Differenz zweier Nachbarn bestimmt und mit der aktuellen minimalen Differenz verglichen. Diese Vergleiche ben�tigen $O(n)$ bis das komplette Array durch ist und somit die minimale Differenz eindeutig bestimmt wurde. \\
Das Indexpaar ist dann a[res], a[res + 1].\\
Die Gesamtlaufzeit ist somit:\\
$$O(nlogn) + O(n) \in O(nlogn)$$
\newpage
 
\subsection*{b)}
 \begin{lstlisting} 
maxFreq (Array a){
	mergeSort (a); // -> O(nlogn)
		int max = 0;
			int counter = 0;
				int res;
					for (int i = 0; i< a.lenght -1; i = i +1){
							if (a[i] == a[i+1] ){
						counter = counter + 1;}
					else {
				if (counter > max){
			max = counter; 
		res = a[i];}
		}
	}
return res;}

 \end{lstlisting}\\

Laufzeit MergSort = $O(nlogn)$\\

\subsection*{c)}


\section*{Aufgabe 2}

\section*{Aufgabe 3}
\subsection*{a)}
Das Pivotelement ist immer der Median und teilt somit immer genau bei der H�lfte. (Siehe Vorlesung 4 Quicksort: Mittlere Analyse) 
\subsection*{b)}
Gesucht ist ein Linearzeitalgorithmus: i H�user mit $1 \leq i \leq n$, H�user haben $p_i \in N$ Personen
\\
1. erstes und n-tes Haus als initiale Indizes b, e w�hlen
\\
2. Anzahl Personen be in Haus bestimmen $\rightarrow$ Abstand $d_b+1-d_b$ bestimmen $\rightarrow$ Anzahl Personen mal den Abstand $\rightarrow$ ergibt $b_s$
\\
Anzahl Personen be $\rightarrow$ Abstand $d_e-d_e-1$ $\rightarrow$ Anzahl Personen mal Abstand $\rightarrow$ ergibt $e_s$
\\
3. Wenn $b_s \leq e_s$, dann $p_b+1 = p_b + p_b+1$ sonst $p_e-1 = p_e + p_e-1$
\\
4. Dann Indexverschiebung: war $b_s<e_s$, dann $b:=b+1, e:=e$ sonst $e:=e-1, b:=b$
\\
5. Sobald b=e terminiert der Algorithmus
\subsection*{c)}
1. Multipliziere niedrigere Personenzahl mit Distanz von A, B. Speicher diesen Wert als Minimum.
\\
2. Falls Summe kleiner als aktuelles Minimum, mache es zu neuem Minimum.
\\
3. Nach Vergleich aller St�dte, gib das Minimum zur�ck



\section*{Aufgabe 4}
\subsection*{a)}
Wahrscheinlichkeit um k mal (hintereinander) Kopf zu werfen ist $p^i$ (analog bei Zahl). Die Wahrscheinlichkeit in festgelegter Reihenfolge beim Werfen k mal Kopf und n-k mal Zahl zu werfen ist also $p^k*(1-p)^n-k$. Die Anzahl an M�glichkeiten, auf die sich die k Erfolge auf n W�rfe verteilen k�nnen ist: $\binom {n} {k}$. Damit macht die Formel Sinn.

\subsection*{b)}


$E (x) = \sum_{x \in M} x \cdot P([X - x)$\\
\newline
 $		  = \sum_{x \in M} x \cdot \binom{n}{x} p^{x} (1-p)^{n-x}$ \\
\newline
$ 		  = \sum_{x = 0}^n x \cdot \binom{n}{x} p^{x} (1-p)^{n-x} \rightarrow x = 0 f�llt aus$\\ 
\newline
$ 		  = \sum_{x = 1}^n x  \cdot \dfrac{n!}{x! (n-x)} p^{x} (1-p)^{n-x}$\\
\newline
$ 		  = n \cdot \sum_{x = 1}^n   \dfrac{n!}{x! (n-x)} p^{x} (1-p)^{n-x}$\\
\newline
$ 		  = n \cdot \sum_{x = 1}^n  \dfrac{(n - 1)!}{(x-1)! \cdot (n-x)!} p^{x} (1-p)^{n-x}$\\
\newline
$ 		  = n \cdot p \cdot \sum_{x = 1}^n  \cdot \binom{n-1}{x-1} \cdot p^{x-1} (1-p)^{(n-1) - (x-1)} \rightarrow x' := x-1$\\
\newline
$ 		  = n \cdot p \cdot \sum_{x^{'} = 0}^{n-1}  \cdot \binom{n-1}{x'} \cdot p^{x'} (1-p)^{(n-1) - x'}$\\
\newline
$ 		  = n \cdot p \cdot (p + (1-p))^{n-1} $ \rightarrow [Binomischer Lehrsatz]\\
\newline
$ 		  = n \cdot p \cdot 1^{n-1} \cdot n \cdot p$ 
\subsection*{c)}
$Var (X) = E(X^{2}) - E(X)^{2}$
\newline
$ = E(X^{2}) - 2E(X)^{2}  + E(X)^{2}$
\newline
$ = E(X^{2}) - 2E(X) \cdot E(X) + E(X)^{2}$
\newline
$ = E(X^{2}) - E(2X) \cdot E(X) + E(X)^{2}$
\newline
$ = E(X^{2}) - E(2XE(X)) + E(X)^{2}$
\newline
$ = E(X^{2} - 2XE(X)) + E(X)^{2}$
\newline
$ = E(X^{2} - 2XE(X) + E(X)^{2})$
\newline
$ = E((X -  E(X))^{2})$
\newline


\end{document}